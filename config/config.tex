
\newcommand{\bmmax}{2}

\usepackage{empheq, amssymb}

\usepackage{lmodern}
\usepackage{amsmath}
\usepackage{amsfonts}
\usepackage{amssymb}
% \usepackage{amsthm,thmtools}

\usepackage{bm} % bold for greek symbols

\usepackage{txfonts}



\usepackage[T1]{fontenc}
\usepackage[utf8]{inputenc} % accept different input encodings

\usepackage[dvipsnames]{xcolor} % defines colors

\usepackage[american]{babel} %  culturally-determined typographical rules

% \usepackage{amsthm}

\usepackage{graphicx}

\usepackage{epstopdf}

%\usepackage{booktabs}
%\usepackage{makecell}
% \usepackage{balance}
%\usepackage{float}

% \restylefloat{table}

\usepackage[noadjust]{cite} % grouped citations



\usepackage{ltxcmds}

% \usepackage{nicefrac}

% \usepackage[export]{adjustbox} % for image frames

\usepackage{mathtools}

% \usepackage{varwidth} % for specifying max widths

\usepackage{subcaption}% for subfigures

\usepackage{multirow} %for tables 

% \usepackage{relsize}

% \usepackage{tkz-linknodes}

% \usepackage{rotating}

\usepackage{blindtext}

% \usepackage[hyphens]{hyperref}
\usepackage{hyperref}

% \usepackage{tikz} % for drawings
% \usetikzlibrary{shapes.geometric, arrows, topaths, shapes,decorations.text, positioning,shadows,trees,shapes.arrows, fadings,arrows.meta,graphs,graphs.standard}


% \usetikzlibrary{external}
% \tikzexternalize[prefix=tikz/]


% \usepackage{algorithm2e} % environment for writing algorithms 

\usepackage{enumitem}

\usepackage{ifthen}

% \usepackage{blkarray}






% \usepackage{showframe}






% \usepackage{nomencl}

\usepackage{etoolbox}

\usepackage{cancel}

\usepackage{empheq}

\usepackage{fontawesome}

% \usepackage{titlesec}

\usepackage{soul}

% \usepackage{flushend}

\usepackage{flushend} % add at the end

% \usepackage{balance}

\usepackage{textcomp}

\usepackage{lipsum}

% \usepackage[thmmarks, amsmath, thref, hyperref, empheq]{ntheorem}


\DeclareMathOperator*{\argmax}{arg\,max}
\DeclareMathOperator*{\argmin}{arg\,min}
\DeclarePairedDelimiter\abs{\lvert}{\rvert}
\DeclarePairedDelimiter\norm{\lVert}{\rVert}
\DeclarePairedDelimiter\bracket{[}{]}
\DeclarePairedDelimiter\paren{(}{)}
\DeclarePairedDelimiter\curl{\lbrace}{\rbrace}
\DeclarePairedDelimiter\ceil{\lceil}{\rceil}
\DeclarePairedDelimiter\floor{\lfloor}{\rfloor}


% create font sizes
\newcommand{\mysmall}{\fontsize{8}{9}\selectfont}
\newcommand{\mySmall}{\fontsize{8.8}{9.5}\selectfont}

% colors
\definecolor{light-gray}{gray}{0.4}
\definecolor{box-gray}{gray}{1}

% editing commands
\newcommand{\xcolor}[1]{\textsl{\textsf{#1}}}
% \newcommand{\xrev}[1]{\textcolor{red}{#1}}
% \newcommand{\xneed}[1]{\textcolor{red}{#1}}
% \newcommand{\xwork}[1]{\textcolor{red}{#1}}
% \newcommand{\xchange}[1]{\textcolor{blue}{#1}}

%
\newcommand{\tran}{^{\mkern-1.5mu\mathsf{T}}}

% url style
\urlstyle{rm}

% hyperlink commands (doi, url, arxiv)
\newcommand{\xdoi}[1]{{doi: \href{https://doi.org/#1}{#1}}\rmFullStop}
\newcommand{\xurl}[1]{{url: \href{#1}{#1}}\rmFullStop}
\newcommand{\xarxiv}[1]{{arXiv:\href{https://arxiv.org/abs/#1}{#1}}\rmFullStop}

% hyperref
\hypersetup{
    unicode=false,          % non-Latin characters in Acrobat’s bookmarks
    pdftoolbar=true,        % show Acrobat’s toolbar?
    pdfmenubar=true,        % show Acrobat’s menu?
    pdffitwindow=false,     % window fit to page when opened
    pdfstartview={FitV},    % fits the width of the page to the window
    pdftitle={Open-Loop Control Co-Design of Floating Offshore Wind Turbines using Linear Parameter-Varying Models},    % title
    pdfauthor={Athul K. Sundarrajan and Yong Hoon Lee and James T. Allison and Daniel R. Herber},     % author
    pdfsubject={},   % subject of the document
    pdfkeywords = {}, % list of keywords
    pdfnewwindow=true,      % links in new window
    colorlinks=true,
    allcolors=blue
}

% command to remove full stop if the next character
% \newcommand*{\rmFullStop}{\rmifnextchar{.}{}{}}

% command to check the next character and replace if present
% \rmifnextchar{X}{[removed text]}{[no X text]}
% if X is the next character, then it is removed and [removed text] is inserted
% otherwise, the character is not removed and [no X text] is inserted
% based on http://tex.stackexchange.com/questions/72827
% \makeatletter
% \newcommand{\rmifnextchar}[3]{%
%   \begingroup
%   \ltx@LocToksA{\endgroup#2}%
%   \ltx@LocToksB{\endgroup#3}%
%   \ltx@ifnextchar{#1}{%
%     \def\next{\the\ltx@LocToksA}%
%     \afterassignment\next
%     \let\scratch= %
%   }{%
%     \the\ltx@LocToksB
%   }%
% }
% \makeatother

\usepackage{nomencl}

\renewcommand\nomgroup[1]{%
  \item[\bfseries
  \ifstrequal{#1}{V}{ Variables}{%
  \ifstrequal{#1}{B}{ Subscripts}{%
  \ifstrequal{#1}{P}{ Notation}{%
  \ifstrequal{#1}{A}{ Acronyms}{}}}}]
}

\renewcommand*{\nompreamble}{\markboth{\nomname}{\nomname}}

\newcommand{\nomdescr}[1]{\parbox[t]{4cm}{\RaggedRight #1}}
\newcommand{\nomwithdim}[5]{\nomenclature[#1]{#2}%
{\nomdescr{#3}\DimensUnits{#4}{#5}}}


\renewcommand{\nomname}{Nomenclature}
\makenomenclature

% https://tex.stackexchange.com/questions/284313/how-do-i-tag-a-subequations-environment-as-a-whole
\makeatletter
\newenvironment{taggedsubequations}[1]
 {%
  % \end{subequations} will advance `equation`
  \addtocounter{equation}{-1}%
  \begin{subequations}%
  % set the current label
  \def\@currentlabel{#1}%
  % redefine \theequation
  \renewcommand{\theequation}{#1.\alph{equation}}%
 }
 {\end{subequations}}
\makeatother %not \makeatletter

\usepackage{dblfloatfix}
\usepackage{float}
% 

\definecolor{block-gray}{gray}{0.95}

% \usepackage{amsthm}
\usepackage[framemethod=TikZ]{mdframed}

\usepackage{xpatch}

\makeatletter
\xpatchcmd{\endmdframed}
  {\aftergroup\endmdf@trivlist\color@endgroup}
  {\endmdf@trivlist\color@endgroup\@doendpe}
  {}{}
\makeatother

% \theoremstyle{definition}
% \theoremseparator{.}

% \theoremheaderfont{\bfseries}
% \theorembodyfont{\mdseries}
% \theoremseparator{.}

% \newmdtheoremenv[
% leftmargin=6pt,%
% rightmargin=0pt,%
% skipbelow=4pt,%
% skipabove=4pt,
% innertopmargin=4pt,%
% innerbottommargin=4pt,
% innerrightmargin=4pt,
% innerleftmargin=4pt,
% linewidth=4pt,
% linecolor=gray,
% topline=false,
% bottomline=false,
% leftline=true,
% rightline=false,
% theoremseparator={:},
% tikzsetting={fill opacity=1},
% ]{definition}%
% {Definition}
% % backgroundcolor=block-gray,%


% \newmdtheoremenv[
% leftmargin=2pt,%
% rightmargin=0pt,%
% backgroundcolor=block-gray,%
% innertopmargin=4pt,%
% innerbottommargin=4pt,
% innerrightmargin=4pt,
% innerleftmargin=4pt,
% skipbelow=30pt,
% linewidth=4pt,
% linecolor=gray,
% topline=false,
% bottomline=false,
% leftline=true,
% rightline=false,
% theoremseparator={:},
% ]{example}%
% {Example}




%
\newcommand{\bz}{\ensuremath{\bm{z}}}
\newcommand{\bx}{\ensuremath{\bm{x}}}

\newcommand{\xa}{\ensuremath{^{\textcolor{black}{a}}}}
% \newcommand{\xa}{\ensuremath{^{\textcolor{black}{(a)}}}}
% \newcommand{\xa}{\ensuremath{^{\textcolor{black}{\circ}}}}
% \newcommand{\xa}{\ensuremath{^{\textcolor{black}{\Join}}}}
% 	\newcommand{\xa}{\ensuremath{^{\textcolor{black}{\boxplus}}}}

\newcommand{\asym}{\ensuremath{\boxplus}}

\newcommand{\xaz}{\ensuremath{_{\textcolor{black}{a_z}}}}

% \newcommand{\xneed}[1]{\textcolor{red}{#1}}


% \usepackage{showframe}

\usepackage{fontawesome}

\usepackage{titlesec}

% DOI and ARXIV Commands for Bib Files
% Written by Daniel Herber
% -----------------------------------------------
% one option is to use the 'note' field with this command
% -----------------------------------------------
% for example, if your doi is 10.2514/1.J052182
% then for the citation for the reference in your bib file, use
% note = "\doi{10.2514/1.J052182}",
% -----------------------------------------------
% for example, if your arxiv number is 0706.1234
% then for the citation for the reference in your bib file, use
% note = "\arxiv{0706.1234}",

% requires hyperref package for \href command
\usepackage{hyperref}

% doi command (use in bib file)
\newcommand{\doi}[1]{{doi:~\href{http://doi.org/#1}{#1}}\rmFullStop}

% arXiv command (use in bib file)
\newcommand{\arxiv}[1]{{arXiv:\href{https://arxiv.org/abs/#1}{#1}}\rmFullStop}

% command to remove full stop if the next character
\newcommand*{\rmFullStop}{\rmifnextchar{.}{}{}}

% command to check the next character and replace if present
% \rmifnextchar{X}{[removed text]}{[no X text]}
% if X is the next character, then it is removed and [removed text] is inserted
% otherwise, the character is not removed and [no X text] is inserted
% based on http://tex.stackexchange.com/questions/72827
\makeatletter
\newcommand{\rmifnextchar}[3]{%
  \begingroup
  \ltx@LocToksA{\endgroup#2}%
  \ltx@LocToksB{\endgroup#3}%
  \ltx@ifnextchar{#1}{%
    \def\next{\the\ltx@LocToksA}%
    \afterassignment\next
    \let\scratch= %
  }{%
    \the\ltx@LocToksB
  }%
}
\makeatother


\usepackage{colortbl}

\definecolor{light-gray}{gray}{0.6}
\newcommand{\grayline}{\arrayrulecolor{light-gray}\hline\arrayrulecolor{black}}

\newcommand{\xsection}[1]{\section[#1]{\MakeUppercase{#1}}}


% local packages
\usepackage{algorithm2e}
\usepackage{xcolor}

%----------------------------------
% custom style commands
%----------------------------------
% variable style command
\newcommand{\xvar}[1]{\textsf{#1}}

% horizontal alignment command
\newcommand{\xvbox}[2]{\makebox[#1][l]{#2}}

%----------------------------------
% set algorithm2e styles
%----------------------------------
% change algorithm font size
\SetAlFnt{\footnotesize}

% change algorithm caption style
\newcommand{\xAlCapSty}[1]{\small\sffamily\bfseries\MakeUppercase{#1}}
\SetAlCapSty{xAlCapSty}

% comment style (algorithms)
% \newcommand{\xCommentSty}[1]{\scriptsize\ttfamily\textcolor{blue}{#1}}
% \definecolor{commentcolor}{HTML}{183E23}
\definecolor{commentcolor}{HTML}{1E4D2B}
\newcommand{\xCommentSty}[1]{\scriptsize\ttfamily\textcolor{commentcolor}{#1}}
\SetCommentSty{xCommentSty}

% change line number style
\newcommand\mynlfont[1]{\scriptsize\sffamily{#1}}
\SetNlSty{mynlfont}{}{} 

 % add the line numbers
\LinesNumbered

% comments right justified
\SetSideCommentRight

% don't print semicolon
\DontPrintSemicolon

% ruled algorithm
\RestyleAlgo{algoruled}

\setlength{\algomargin}{1.1em}
% Two Column Compatible, Variable Width Algorithm Environments
% Written by Daniel Herber
% Based on modifications to the stack exchange answer at
% http://tex.stackexchange.com/questions/23296/setting-caption-rule-width-in-algorithm2e-algs/39574

% inputs
% 1 : float specifiers (htbp!)
% 2 : algorithm width (length)
% 3 : indent width (length)

% column environment
% \begin{vAlgorithm}[float specifiers]{algorithm width}{indent width}
% \end{vAlgorithm}

% page environment
% \begin{vAlgorithm*}[float specifiers]{algorithm width}{indent width}
% \end{vAlgorithm*}

% required packages
\usepackage{algorithm2e}
\usepackage{etoolbox}
\usepackage{xstring}
\usepackage{calc}
 
% define lengths
\newlength{\xalgowidth}
\newlength{\xalgoremainder}
\newlength{\xindentwidth}

% define vAlgorithm* environment
\newenvironment{vAlgorithm*}[3][]{% before
  \setlength{\xalgowidth}{#2} % set algorithm width from second input
  \setlength{\xindentwidth}{#3} % set indent width from third input
  \setlength{\xalgoremainder}{\textwidth-\xalgowidth} % calculate indent to center the float
  \SetCustomAlgoRuledWidth{\xalgowidth} % set the rule width
  \IncMargin{\xindentwidth}
  \begin{algorithm*}[#1]
}% end before
{% after
  \end{algorithm*} 
  \DecMargin{\xindentwidth}
}% end after

% define vAlgorithm environment
\newenvironment{vAlgorithm}[3][]{% before
  \setlength{\xalgowidth}{#2} % set algorithm width from second input
  \setlength{\xindentwidth}{#3} % set indent width from third input
  \setlength{\xalgoremainder}{\columnwidth-\xalgowidth} % calculate indent to center the float
  \SetCustomAlgoRuledWidth{\xalgowidth} % set the rule width
  \IncMargin{\xindentwidth}
  \begin{algorithm}[#1] % column
}% end before
{% after
  \end{algorithm} % column
  \DecMargin{\xindentwidth}
}% end after

% redefine algorithm commands
\makeatletter
\patchcmd{\@algocf@start}{%
\begin{lrbox}{\algocf@algobox}%
}{%
\rule{0.5\xalgoremainder}{\z@}% indent
\begin{lrbox}{\algocf@algobox}%
\begin{minipage}{\xalgowidth}%
}{}{}
\patchcmd{\@algocf@finish}{%
\end{lrbox}%
}{%
\end{minipage}%
\end{lrbox}%
}{}{}
\makeatother


\usepackage{accents}


\newcommand{\pmin}{\underaccent{\bar}{\bm{P}}}
\newcommand{\pmax}{\bar{\bm{P}}}

\newcommand{\rmin}{\underaccent{\bar}{\bm{R}}}
\newcommand{\rmax}{\bar{\bm{R}}}

\newcommand{\tpmin}{\underaccent{\bar}{T}_p}
\newcommand{\tpmax}{\bar{T}_p}

\newcommand{\trmin}{\underaccent{\bar}{T}_r}
\newcommand{\trmax}{\bar{T}_r}

\newcommand{\xub}[1]{\underaccent{\bar}{#1}}
\newcommand{\xob}[1]{\bar{#1}}

\newcommand{\gline}{\arrayrulecolor{light-gray}\hline\arrayrulecolor{black}}

\definecolor{needcolor}{HTML}{C62828}
% \newcommand{\xneed}[1]{\textcolor{needcolor}{#1}}
% \newcommand{\xrev}[1]{\textcolor{red}{#1}}
% \newcommand{\xchecked}[1]{\textcolor{needcolor}{#1}}

\newtheorem{theorem}{Theorem}
\newtheorem{definition}{Definition}

\renewcommand{\qed}{\hfill\ensuremath{\blacksquare}}

\newcommand{\xtran}{\ensuremath{^\mathsf{T}}}
% \newcommand{\xtran}{\ensuremath{^\prime}}

\newcommand{\openFAST}{OpenFAST}
