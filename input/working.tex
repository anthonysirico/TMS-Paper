\xsection{Basic outline}
Multiphysics software performing high-fidelity modeling and simulations are the primary tools used for design optimization of floating offshore wind turbines and their controllers.
Early stage design studies simply look to understand the properties and  behaviour of the system, to create a framework that can help design the system. 
For early stage design studies, the use of high-fidelity modelling tools is counter productive, as such tools are computationally expensive and time consuming.
To mitigate this drawback, reduced order models are developed by designers, that capture just the essential physics of the system. 
Results from these lower order models are first validated against the simulations from high-fidelity tools.
After validation, thee models are then linearized around set point values and used for controller design.
Well-established methods from linear systems theory, can then be used to understand the behaviour of these linear models.

Another way to mitigate the expense is to obtain linearized models directly from these high-fidelity modelling tools. 
A major drawback with these linearized models is that these models become inaccurate when the system moves away from the operating points.
Due to these inaccuracies, these models by themselves, cannot be used extensively for early stage design studies.
In this work, we discuss the use of Linear Parameter Varying (LPV) models to overcome the drawbacks with these linearized model.
We show that the new LPV models can be used to create accurate open loop optimal control trajectories.
